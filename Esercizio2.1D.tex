\subsection{Esercizio 1 Daboni Wedlin}
Gli eventi $E_{1}, E_{2},  E_{3}, E_{4}, E_{5}$ sono giudicati scambiabili.
Sono assegnate le seguenti probabilità:
\begin{itemize}[label=]
	\item $P(E_{2})=\frac{1}{2}$
	\item $P(E_{3}\wedge E_{5})=\frac{1}{4},$ 
	\item $P(E_{1}\wedge\overline{E}_2\wedge E_{3}\wedge\overline{E}_{4}\wedge E_{5})=P(E_{1}\wedge\overline{E}_{2}\wedge\overline{E}_{3}\wedge\overline{E}_{4}\wedge\overline{E}_{5})=P(E_{1}\wedge E_{2}\wedge E_{3}\wedge E_{4}\wedge E_{5})=\frac{1}{30}$.
\end{itemize}

Si calcolino:
\begin{itemize}[label=]
	\item $P(E_{2}\wedge E_{3}\wedge E_{4})$
	\item $P(E_{1}\wedge E_{2}\wedge E_{3}\wedge E_{4})$
	\item $P(E_{1}\wedge E_{2}\wedge\overline{E}_{3}\wedge\overline{E}_{4}\wedge\overline{E}_{5})$
\end{itemize}

\textbf{Svolgimento}:
\\
\bigskip

Il caso è quello di un \textit{processo di alternativa semplice limitato} con 5 eventi ($E_{1}, E_{2},  E_{3}, E_{4}, E_{5}$), ai quali corrispondono 5 variabili aleatorie \\
($X_{1}, X_{2},  X_{3}, X_{4}, X_{5}$) sotto l’ipotesi di scambiabilità. 
Si parla di processo di alternativa semplice limitato poichè gli eventi in questione sono di numero limitato, di tipo elementare (vero-falso) e le variabili indicatrici associate sono di tipo 0-1. \\
L’ipotesi di scambiabilità garantisce che dati \textit{n} eventi, non necessariamente indipendenti, la probabilità che se ne realizzino esattamente \textit{h} su \textit{n} non è dipendente dall'ordine degli eventi stessi ovvero, permutando l’ordine delle variabili indicatrici, la probabilità della loro realizzazione resta immutata. \\

Usando la terminologia del libro indicheremo con:
\begin{itemize}[label=-]
    \item $\omega^n_h$ la probabilità che dati \textit{n} eventi se ne realizzino esattamente \textit{h}, indipendentemente dal loro ordine
     \item $\frac{\omega^n_h}{\binom{n}{h}}$ la probabilità di una singola \textit{traiettoria} formata da una determinata sequenza di \textit{h} successi su \textit{n} prove. Infatti di tali possibili traiettorie ne avremo $\binom{n}{h}$.
    \item $\omega_h$ la probabilità di \textit{h} successi su \textit{h} prove.
\end{itemize}


Riscriviamo i dati del nostro problema nel seguente modo:
\begin{align*}
	P(E_{2})=\omega_{1}&=\frac{1}{2}\\
	 P(E_{3}\wedge E_{5}) =\omega_{2} &=\frac{1}{4}\\
	 P(E_{1}\wedge\overline{E}_{2}\wedge E_{3}\wedge\overline{E}_{4}\wedge E_{5}) =\frac{\omega_{3}^5}{\binom{5}{3}}&=\frac{1}{30} \\
	 P(E_{1}\wedge\overline{E}_{2}\wedge\overline{E}_{3}\wedge\overline{E}_{4}\wedge\overline{E}_{5}) =\frac{\omega_{1}^5}{\binom{5}{1}}&=\frac{1}{30} \\
	 P(E_{1}\wedge E_{2}\wedge E_{3}\wedge E_{4}\wedge E_{5}) =\omega_{5}&=\frac{1}{30}.
\end{align*}
Ciò che dobbiamo calcolare sarà quindi:
\begin{align*}
 P(E_{2}\wedge E_{3}\wedge E_{4})&=\omega_{3}\\
 P(E_{1}\wedge E_{2}\wedge E_{3}\wedge E_{4})&=\omega_{4}\\
 P(E_{1}\wedge E_{2}\wedge\overline{E}_{3}\wedge\overline{E}_{4}\wedge\overline{E}_{5})&=\frac{\omega_{2}^5}{\binom{5}{2}}
\end{align*}

Dalla teoria dei processi scambiabili sappiamo che le $\omega_h$ e le $\omega^n_h$ sono
legate da specifiche relazioni e che è possibile ricavare le une dalle altre. 
Un modo per ottenere le serie richieste dai dati forniti è quello di sfruttare la 
seguente equazione:

\begin{align*}
    \omega_h^n = \binom{n}{h}(-1)^{n-h}\cdot\Delta^{n-h}\cdot\omega_h \quad \quad \quad  h\leq n
\end{align*}

Possiamo quindi riscrivere le quantità richieste e date dal problema  in funzione 
delle $\omega_h$:
\begin{align*}
    \frac{\omega_2^5}{\binom{5}{2}} &= (-1)^3\Delta^3\omega_2=-(\omega_5-3\omega_4+3\omega3-\omega_2)\\
    \frac{\omega_1^5}{\binom{5}{1}} &= (-1)^4\Delta^4\omega_1=\omega_5-4\omega_4+6\omega3-4\omega_2+\omega_1 =  -4\omega_4+6\omega_3-\frac{7}{15} &= \frac{1}{30} \\
    \frac{\omega_3^5}{\binom{5}{3}} &= (-1)^2\Delta^2\omega_3=\omega_5-2\omega_4+\omega_3=-2\omega_4+\omega_3+\frac{1}{30} &= \frac{1}{30} 
\end{align*}

Mettendo a sistema le ultime due equazioni potremo ricavare quanto segue:
\begin{align*}
\begin{cases} 
-4\omega_4+6\omega_3-\frac{7}{15}= \frac{1}{30} \\
-2\omega_4+\omega_3+\frac{1}{30} = \frac{1}{30} 
\end{cases}
\Longrightarrow
\begin{cases} 
\omega_3 = \frac{1}{8}\\
\omega_4 = \frac{1}{16}
\end{cases}\\ \\
\frac{\omega_2^5}{\binom{5}{2}} = -\frac{1}{30}+\frac{3}{16}-\frac{3}{8}+\frac{1}{16} = \frac{7}{240}
\end{align*}

In conclusione, le probabilità richieste saranno:
\begin{align*}
	 P(E_2\wedge E_3\wedge E_4) &= \omega_3 = \frac{1}{8}\\
	 P(E_1\wedge E_2\wedge E_3\wedge E_4)&= \omega_4 = \frac{1}{16}\\
	 P(E_1\wedge E_2\wedge\overline{E}_3\wedge\overline{E}_4\wedge\overline{E}_5) &= \frac{\omega_2^5}{\binom{5}{2}} = \frac{7}{240}
\end{align*}

